\section{Entwurfsprinzipien}
Essenz jedes Musters kurz zusammengefasst. 

\paragraph{Entwurfsprinzipien}

\begin{itemize}[leftmargin=0.2in]
	\item Identifizieren Sie die Aspekte Ihrer Anwendung, die sich aendern koennen und trennen Sie sie 
  von denen, die konstant bleiben.
	\item Programmieren Sie auf eine Schnittstelle, nicht auf eine Implementierung.
	\item Ziehen Sie die Komposition der Vererbung vor.
	\item Streben Sie bei Entw"urfen mit interargierenden Objekten nach lockerer Kopplung (einfachere 
Erweiterung moeglich).
	\item Klassen sollten fuer Erweiterung offen, aber fuer Veraenderung geschlossen sein. (Mit Vorsicht zu 
betrachten, dies fuehrt zu hoeherer Komplexit"at und sollte nur in sinnvollen Faellen 
angewendet werden.)
	\item St"utzen Sie sich auf Abstraktionen. St"utzen Sie sich nicht auf konkrete Klassen.  
	\item Principle of Least Knowledge - talk only to your immediate friends.
\end{itemize}

