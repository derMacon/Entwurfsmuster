\documentclass[a4paper, 11pt]{scrartcl}

%Packages fuer die Darstellung von Quellcode
\usepackage{beramono} %schoene Schriftart
\usepackage{color}
\usepackage[dvipsnames]{xcolor}
\usepackage{listings}

\usepackage[utf8]{inputenc}
\usepackage[english, ngerman]{babel}
\usepackage[T1]{fontenc}
\usepackage{lmodern}

\usepackage{enumitem}
\usepackage[a4paper, bottom=2cm,includeheadfoot]{geometry}
\usepackage{afterpage} %Um Titelseite mit gesonderter Formatierung zu belegen
\usepackage{blindtext}
\usepackage{todonotes}

% Noetig fuer verschachtelte Imports 
\usepackage{import}
\usepackage{subcaption}

% Bilder
\usepackage{graphicx}

% Header Informationen festgelegt
\usepackage{fancyhdr}
\pagestyle{fancy}
\fancyhf{}
\rhead{\rightmark}
\chead{\thepart}
\lhead{\nouppercase{\leftmark}}
\cfoot{\thepage}


%Quellcodestyle Spezifikationen
\definecolor{DarkPurple}{rgb}{0.4,0.1,0.4}
\definecolor{DarkCyan}{rgb}{0.0,0.5,0.4}
\definecolor{LightLime}{rgb}{0.3,0.5,0.4}
\definecolor{Blue}{rgb}{0.0,0.0,1.0}

% Macro um nicht jedes Mal die listing und namen-Referenz angeben zu muessen
\newcommand{\lstref}[1]{(siehe Listing \ref{#1}, S. \pageref{#1}, \nameref{#1})}


% Syntaxhighlighting festgelegt
\lstdefinestyle{CodeHighlighting} 
{
language=Java, %mit mehreren Sprachen moeglich, ermoeglicht Syntaxhighlighting
columns=flexible,
numbers=left,
frame=single,
frameround=tttt,
showstringspaces=false,
basicstyle=\footnotesize\ttfamily,
keywordstyle=\bfseries\color{DarkPurple},
commentstyle=\itshape\color{LightLime},
stringstyle=\color{Blue}
}

% Titelseite
\title{Entwurfsmuster - von Kopf bis Fuß}
\subtitle{\Large{Zusammenfassung}}
\date{} %Datum unerwuenscht

\begin{document}
\newgeometry{a4paper, bottom=5cm}
\maketitle
\pagestyle{plain}
\tableofcontents
\clearpage
\pagestyle{fancy}

\section{Entwurfsprinzipien}
Essenz jedes Musters kurz zusammengefasst. 

\begin{itemize}[leftmargin=0.2in]
	\item Entwurfsprinzipien:
	\item Identifizieren Sie die Aspekte Ihrer Anwendung, die sich aendern koennen und trennen Sie sie 
  von denen, die konstant bleiben.
	\item Programmieren Sie auf eine Schnittstelle, nicht auf eine Implementierung.
	\item Ziehen Sie die Komposition der Vererbung vor.
	\item Streben Sie bei Entwuerfen mit interargierenden Objekten nach lockerer Kopplung (einfachere 
Erweiterung moeglich).
	\item Klassen sollten fuer Erweiterung offen, aber fuer Veraenderung geschlossen sein. (Mit Vorsicht zu 
betrachten, dies fuehrt zu hoeherer Komplexitaet und sollte nur in sinnvollen Faellen 
angewendet werden.)
	\item Stuetzen Sie sich auf Abstraktionen. Stuetzen Sie sich nicht auf konkrete Klassen.  
\end{itemize}

	







\end{document}